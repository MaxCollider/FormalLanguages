\documentclass[12pt,a4paper]{extreport}
\usepackage[l2tabu,orthodox]{nag}
\usepackage[]{float}
\usepackage{enumitem}

% Пожалуйста, не меняйте указанные ниже установки полей в документе
\usepackage[left=10mm,right=50mm, top=15mm,bottom=15mm,bindingoffset=0cm]{geometry}

\usepackage{amssymb,amsmath,amsthm}

\usepackage{fontspec}

\setmainfont[Ligatures=TeX]{STIX}
\newfontfamily{\cyrillicfont}[Ligatures=TeX]{STIX}
\setmonofont{Fira Mono}
\newfontfamily{\cyrillicfonttt}{Fira Mono}

\usepackage{polyglossia}
\setdefaultlanguage{russian}
\setotherlanguage{english}


\title{Описание алгоритма}
\author{Вдовин Максим}

\begin{document}
\maketitle

\par Дано $\alpha$ и слово $u \in \{a, b, c\}*$. Нужно найти длину наибольшего префикса $u$, принадлежащего $L$.

\par Решать эту задучу буду методом динамического программирования. Пусть у нас есть строки: $reg$ - регулярное выражение, $str$ - слово, для которого нужно найти наибольший общий префикс принадлежащий языку. Также пусть длинна $|reg| = m$, а $|str| = n$. Суть алгоритма в том, чтобы для каждого символа регулярки понять какую часть слова она распознаёт. Заведем массив булей $dp[l][i][j]$, где $l$ - номер текущего символа из $reg$, $i$ и $j$ - левая и правая граница, той части слова, которая считывается регулярным выражением. Значение $true$ в $dp[l][i][j]$ означает, что $reg[l]$ принимает в себя подслово $[i,j)$.
\par Далее я покажу как подсчитать $dp$ в зависимости от разных случев $reg[l]$
\begin{enumerate}
    \item Если $reg[l]$ ~--- $1$, то $\forall i \quad dp[l][i][i] = true$.
    \par Очевидно, что таким образом мы задаём пустое слово.
    
    \item Если $reg[l]$ ~--- буква, то $\forall i$, если $(str[i] == reg[l])$ мы записываем $dp[l][i][i + 1] = true$
    
    \item Если $reg[l] = "+"$, то мы будем искать такие $l_0$ и $l_1$, что они являются операндами для $reg[l]$. Далее подсчитаем $\forall i, j \quad dp[l][i][j] = dp[l_0][i][j] \, || \, dp[l_1][i][j]$
    
    \item Если $reg[l] = "."$,то мы будем искать такие $l_0$ и $l_1$, что они являются операндами для $reg[l]$. Далее подсчитаем возьмем некоторый разделитель ~--- $pivot$, который будет изменяться $\forall i,j$, так что $i \leq pivot \leq j$. Массив $dp$ теперь будет считаться по следующей формуле: \[\forall i, j \quad dp[l][i][j] = dp[l][i][j] \, || \, (dp[l_0][i][pivot] \, || \, dp[l_1][pivot][j])\].
    
    \par Здесь мы как бы говорим, что подслово $str[i..j)$ может быть получено из регулярного выражения, соответствующего $reg[l]$, если существует разбиение этого слова на два слова: $str[i..j) = str[i...pivot)str[pivot...j)$ и каждое из этих двух слов может быть получено из соответствующих операндов для $reg[l]$.
    
    \item Если $reg[l] = "*"$,то мы будем искать такое $l$, что оно является операндом для $reg[l]$. Далее подсчитаем возьмем некоторый разделитель ~--- $pivot$, который будет изменяться $\forall i,j$, так что $i \leq pivot \leq j$. Массив $dp$ теперь будет считаться по следующей формуле: \[\forall i, j \quad dp[l][i][j] = dp[l][i][j] \, || \, (dp[l_0][i][pivot] \, || \, dp[l_1][pivot][j])\]
    
    \par Здесь мы как бы говорим, что подслово $str[i...j)$ может быть получено из регулярного выражения для $reg[l]$, если существует разбиение этого слова на несколько слов: $str[i...j) = str[i...i_1)str[i_1...i_2) \dots str[i_k...j)$ и каждое из этих слов может быть получено из регулярного выражения для $reg[l_0]$. Это в свою очередь эквивалентно тому, что существует разбиение слова $str[i...j)$ на два слова: $str[i...j) = str[i...i_l)str[i_l...j)$ и первое слово может быть получено из регулярного выражения для $reg[l]$, а второе из РВ для $reg[l_0]$.
\end{enumerate}
\end{document}
